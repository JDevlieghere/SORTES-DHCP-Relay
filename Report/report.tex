\documentclass[11pt]{article}
\usepackage[english]{babel}
\usepackage{listings}
\usepackage{xcolor}
\usepackage{courier}
\usepackage{hyperref}

\lstset{
language=sh
,breaklines=true
,frame=single
,tabsize=2
,basicstyle=\ttfamily
,showstringspaces=false
,numbers=left
,numberstyle=\tiny\color{gray}}

\setlength\parindent{0pt}

\newcommand{\io}[1]{\textbf{#1}}

\title{SORTES - DHCP Relay}
\author{Dieter Castel (0256149) \\ Jonas Devlieghere (0256709)}
\date{\today}

\begin{document}
\maketitle
\newpage

\tableofcontents
\newpage

\section{DHCP Relay}
Assuming the reader is familiar with the Dynamic Host Configuration Protocol
We'll explain in short how the DHCP Relay works and focus on the four stages
listed below. The purpose of a DHCP Relay is allowing clients from a different
subnet than that of the DHCP to still communicate with the server. Because
broadcasts are limited to the local subnet, the relay forwards the broadcast
using unicast to the DHCP server. Messages from the server are returned to the
relay, also using unicast, which are than forwarded to the client.

\begin{itemize}
	\item \textbf{DHCP Discovery}: The client broadcasts on his subnet to
	discover DHCP servers. The Relay will forward this message to the DHCP
	server using unicast.
	\item \textbf{DHCP Offer}: The server replies to the client by offering an
	IP. This message is forwarded by the relay from the server to the client.
	\item \textbf{DHCP Request}: The client will confirm an offer using a
	request message. This is once again forwarded to the server by the relay.
	\item \textbf{DHCP Acknowledgement}: The acknowledgement finally confirms
	the deal between the client and the server. This message is forwarded from
	the server to the client by the relay.
\end{itemize}

\section{DHCP Relay Design}
The relay consists of 3 large parts. The first one is listening for incoming UDP
packets and filtering those related to DHCP. The second and third part consists
of forwarding messages to the server and client respectively.

\subsection{Listening for DHCP Packets}
The relay is constantly listening for DHCP packets from both the server and the
client. Once a packet is received, the appropriate relay function is called and
the packet forwarded.

\subsection{Forwarding to Server}
When forwarding a packet to the server, the content is copied and the relay's IP
is set in the \texttt{GIADDR} field. This field is used by the server to
determine the correct subnet and as destination address for the responding
packet. An ARP request is made to determine the servers MAC-address.\\
\\
\textbf{Note:} Usually the first packet forwards fails due to an unfinished ARP
request. By the time the packet is received again, the request has been
fulfilled and the packet is forwarded as expected. We chose not to introduce a
delay since we are certain a duplicate of the packet will be received soon.

\subsection{Forwarding to Client}
Forwarding a DHCP packet from the server to the client is even more
straightforward. The whole packet is simply broadcasted onto the local subnet.

\section{Deployment}

\subsection{Building}
Simply run deploy.sh to build and deploy the application to the PIC. You will
automatically enter the tftp shell allowing you to copy the hex file to the
board.

\subsection{Testing}
The following setup was used to test the relay:

\begin{itemize}
	\item The PIC connected to port 2 (LAN) of the router. The relay has a
	static IP of \texttt{192.168.97.60}.
	\item The DHCP server is connected to port 1 (WAN) of the router. This port
	is assigned the ip \texttt{192.168.2.1} and the server itself is set to the
	static IP \texttt{192.168.2.2}.
	\item The client connected to a LAN port of the router.
\end{itemize}

\subsubsection{Router Configuration}
We had to configure the router to allow our desired setup. First of all we
disabled the internal DHCP server. We then assigned a static IP to the WAN port
(\texttt{192.168.2.1}) and disabled the NAT. \\
\\
\textbf{Note:} Without the DHCP running, you will have to use a static IP on the
client of server machine to access the router configuration web page or upload a
new software version to the PIC. This might be a hassle, especially when
switching between a static IP and one assigned by DHCP when testing.

\subsubsection{Validation}
We ran Wireshark on both the computer acting as DHCP server and the one acting
as client. This allowed us to capture the packages received and sent by both
devices.

\section{Difficulties \& Particularities}
The hardest part of the assignment was understanding the TCP/IP stack and its
implementation. The manual did provide some insight but lacked a proper
explanation of how to add a new component such as the relay. Most part of the
solution is therefore based on the existing implementation, adapted to fit our
needs.
\\\\
Most problems were caused by a lack of knowledge of the stack API. Especially
when rewriting code, this created a lot of confusion. For example, we based our
DHCP listening function on the one used by the available DHCP server. This code
ignored some critical bytes needed for receiving packets from the server. It
took us a long time to discover this problem.
\\\\
The experience we had with the PIC, the compiler and the hardware interface,
which all caused us a lot of troubles last time, significantly improved our
working proficiency.

\end{document}
